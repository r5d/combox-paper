\begin{abstractpage}
  File storage providers on the Internet have made it non-trivial for
  individuals to store personal files on the file storage provider's
  computers. After Mr. Snowden disclosed information about the
  National Security Agency' (NSA) surveillance programs that allowed
  the NSA to access information stored on file storage provider'
  computers, online file storage became a non-solution for storing
  personal files for everyone who detested the possibility of somebody
  else being able to access their personal files. In the past, there
  have been separate efforts to come with a solution to allow
  individuals to use storage space provided by file storage providers
  in a way that it made it impossible for file storage providers and
  to access the files. combox is one such effort. It allows an
  individual to store personal files in the ``combox directory'' on
  all her computers (running GNU/Linux or OS X) and the combox program
  takes the files, splits and encrypts them and spreads them across
  file storage providers' directories. Therefore, when an individual
  uses storage space provided by file storage providers through
  combox, each file storage provider gets only a part of the file in
  an encrypted form.
\end{abstractpage}
