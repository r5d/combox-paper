\chapter{Introduction}

\epigraph{From a security perspective, if you're connected, you're
  screwed.}{\textit{Daniel J. Bernstein}}

Internet companies have made it trivial for computer users to store
data/information on their computers and at the same time there is a
lot of evidence of governments and other powerful organizations being
able to access information/data stored on the Internet companies'
computers\cite{website:wikileaks-spyfiles}. Also most companies add a
standard clause in their privacy policy that allows them to disclose
information about users or information stored/created by users to
``third parties'':

\begin{quote}
  \emph{Law \& Order}. We may disclose your information to third
  parties if we determine that such disclosure is reasonably necessary
  to (a) comply with the law; (b) protect any person from death or
  serious bodily injury; (c) prevent fraud or abuse of Dropbox or our
  users; or (d) protect Dropbox's property rights. -- Dropox Privacy
  Policy\cite{website:dropbox-privacy}
\end{quote}

In this type of world, it did be good to have a program that would
encrypt all the data/information before storing it on the storage
provided by Internet companies. combox aims to be one such program
which not only encrypts but stores only a part of the encrypted
data/information on the Internet company' storage, thus making it
non-trivial for ``third parties'' get access the user's
data/information. Section \ref{1-sec-b} gives a conceptual
introduction to combox; Section \ref{1-sec-cb-diff} enumerates how combox
is different from Combo-Box; lastly, section \ref{1-sec-using-cb}
contains information on how one can start using combox.

\section{What is combox?}\label{1-sec-cb}

combox allows the user to store all her files in the ``combox
directory'' and combox picks each file stored in the combox directory,
splits them into N shards, encrypts each of the N shards and spreads
the shards to N node directories. A ``node directory'' is the
directory of the file storage provider (Dropbox directory is a node
directory). Figure \ref{fig:1-combox-overview-0}, illustrates how a file
called \verb+strunk-white.pdf+ is split, encrypted and spreaded across
N node directories; shards \verb+strunk-white.pdf.shard0+ to
\verb+strunk-white.pdf.shardN+ are encrypted.

\begin{figure}[h]
\begin{verbatim}

                                  __________________________
                                  |                         |
                               -->| strunk-white.pdf.shard0 |
                               |  |                         |
         ___________________   |  |_________________________|
         |                  |  |  node directory 0
         | strunk-white.pdf | /
         |                  | |   __________________________
         |__________________| |\  |                         |
         combox directory     ||  | strunk-white.pdf.shard1 |
                              ||->|                         |
                              |   |_________________________|
                              |   node directory 1
                              |           .
                              |           .
                              |           .
                              |
                              |   __________________________
                              |   |                         |
                              --->| strunk-white.pdf.shardN |
                                  |                         |
                                  |_________________________|
                                  node directory N
\end{verbatim}
\caption{combox overview - file splitting}
\label{fig:1-combox-overview-0}
\end{figure}

combox does not sync encrypted shards stored in the node directories
to the respective file storage provider's server and it depends on the
respective file storage provider's client program to sync the
shards.

combox can be used on all of the user's computers. For instance, the
user can install combox on her second computer and combox will
reconstruct the file from the encrypted shards stored in the node
directories into the combox directory; figure
\ref{fig:1-combox-overview-1} illustrates this. Here too, combox
depends on the client program of the respective file storage provider
to sync shards to/from the file storage provider's server to/from the
respective node directory on the user's computer.

\begin{figure}[h]
\begin{verbatim}

       __________________________
       |                         |
       | strunk-white.pdf.shard0 |
       |                         |\
       |_________________________| \   ___________________
       node directory 0             \  |                  |
                                    |->| strunk-white.pdf |
       __________________________  |-->|                  |
       |                         | | ->|__________________|
       | strunk-white.pdf.shard1 |-- | combox directory
       |                         |   |
       |_________________________|   |
       node directory 1              |
               .                     |
               .                     |
               .                     |
                                     |
       __________________________    |
       |                         |   |
       | strunk-white.pdf.shardN |----
       |                         |
       |_________________________|
       node directory N

\end{verbatim}
\caption{combox overview - file reconstruction}
\label{fig:1-combox-overview-1}
\end{figure}

As of combox \verb+v0.2.2+, combox is compatible on GNU/Linux and OS
X, it supports just two file storage providers -- Google Drive and
Dropbox.

\section{How is combox different from Combo-Box?}\label{1-sec-cb-diff}

\section{Using combox}\label{1-sec-using-cb}

Installing and running combox is relatively easy for Unix users:

\begin{verbatim}
   $ pip install combox
   $ combox
\end{verbatim}

For detailed information on installing combox, see
https://ricketyspace.net/combox/setup/.

\subsection{Caveats}

combox is extremely event-driven and depends on file-system events to
do the right thing when a file is created/modified/moved/deleted, so
the user must sure to start combox before starting the file storage
providers' client programs that sync encrypted shards to the
respective node directories; on most GNU/Linux distributions this can
be automated through by using the distribution's startup system (most
GNU/Linux distributions seem to use
\verb+systemd+\cite{website:systemd} these days).
