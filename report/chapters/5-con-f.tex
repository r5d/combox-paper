\chapter{Conclusion and Future Work}\label{ch:5}

\epigraph{The answer to this is very simple. It was a joke. It had to
  be a number, an ordinary, smallish number, and I chose that
  one. Binary representations, base thirteen, Tibetan monks are all
  complete nonsense. I sat at my desk, stared into the garden and
  thought '42 will do' I typed it out. End of story.
}{\textit{Douglas Adams, November 1993}}

combox is at a stage where it can be used by users as a tool to use
the storage provided by two file storage providers -- Google Drive and
Dropbox -- such that only part of each file in the encrypted form is
stored on the computers of the file storage providers; this method of
storing files on file storage providers makes it difficult but not
impossible for ``third parties'' to gain access to the user's personal
files.

combox is at version 0.2.2, it is a python package licensed under the
GNU General Public License version 3 or later. It is compatible with
GNU/Linux and OS X. The program is considered to be in ``alpha'' stage
and must be used for experimental use only, it is not recommended to
store critical files on storage provided by file storage providers
using combox. Individuals who wish to try combox would want to look at
\url{https://ricketyspace.net/combox/setup/} to get the program
installed on their machines; Individuals who want to hack/learn about
combox would want to look at
\url{https://ricketyspace.net/combox/api/}. combox's canonical source
repository is at \url{https://git.ricketyspace.net/combox}, the
repository is also mirrored at
\url{https://bitbucket.org/bgsucodeloverslab/combox/src} and
\url{http://rsiddharth.ninth.su/git/cb.git/}.

There are a lot of things that can be done to improve combox, what
follows is a non-exhaustive list of things to do in the future:

\begin{itemize}
\item Make combox cognizant about space available on each node
  directory. At the moment, combox reads the amount of free space
  available on each node directory (file storage provider's directory)
  when configuring combox on a computer but does not use this
  information to reckon the space left in each node directory.
\item Re-think \verb+combox.events+ module. This module was written
  with the assumption that combox will be the only one to make changes
  to the node directories. This assumption was found to be not true
  when manually testing combox with node clients (Google Drive and
  Dropbox client that sync files to/from the respective node
  directories to/from their respective servers); both the Google Drive
  and the Dropbox client make modifications to the Google Drive and
  Dropbox directory respectively whenever pulling a modified shard
  from their server to this computer, this behavior broke combox and
  major changes were made to the \verb+combox.events+ module to make
  it understand the node client's behavior in the node directory;
  these changes, increased the complexity of the classes defined in
  the \verb+combox.events+; it would be great to re-think this module
  in such a way that it reduces its complexity.
\item Evaluate if more information needs to tracked about each file in
  the combox directory; at the moment, combox only keeps track of the
  SHA256 hash of each file stored in the combox directory.
\item Support more file storage providers; for this, ideally no code
  needs to be written for supporting a new file storage provider,
  combox must be tested with the new file storage provider's directory
  as a node directory. If the new file storage provider's client (that
  sync's the shards their servers) makes non-standard changes to its
  directory (like the official Dropbox and Google Drive clients do),
  then the \verb+combox.events.NodeDirMonitor+ must be accordingly
  updated to make combox cognizant about the file storage provider
  client's non-standard behavior.
\item Make unit tests more modular. At the moment, there are some unit
  test functions that test more than one usecase/facet of a function
  or class; for instance, the \verb+test_CDM+ test method part of the
  the \verb+tests.events_test.TestEvents+ test class tests the
  correctness of the \verb+combox.events.ComboxDirMonitor+ for file
  creation, deletion, rename and modification; this method would
  ideally broken down into four tests methods.
\item Make combox Python 3 compatible. The \verb+2to3+ program (which
  is part of the standard Python library since Python version 2.6) and
  the \verb+six+ library can be used to achieve this. See Appendix
  \ref{a-python3c} for more information on this.
\item Support Microsoft Windows. The way to make combox compatible
  with Windows will be to run unit tests on Windows, the failing tests
  might give pointers to what parts of combox needs to be changed in
  order for it to be compatible with Windows. Individuals interested
  in making combox compatible with Windows might find
  \url{https://ricketyspace.net/combox/setup/#windows} useful; it
  contains information about setting up the development environment
  for combox on Windows.
\end{itemize}

