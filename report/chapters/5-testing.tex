\chapter{Testing}

\section{Unit testing}

The \verb+nose+\cite{pylib:nose} testing framework was used to
write unit tests for the functions and classes part of the
\verb+combox.config+, \verb+combox.crypto+, \verb+combox.events+,
\verb+combox.file+, \verb+combox.silo+ \verb+combox._version+
modules. Unit tests were not written for \verb+combox.cbox+,
\verb+combox.gui+, \verb+combox.combox.log+ modules.

Unit tests for combox become reality by pure serendipity. During the
time, when I started working on combox, I was learning to use the
\verb+nose+ library to unit test python code. Since, \verb+combox+ was
being written in python, I started making it a norm to write unit
tests for functions and classes in combox modules.

As mentioned before, unit tests were not written for some modules
either because it would make no sense to write one (for the
\verb+combox.cbox+ module, for instance, which basically uses
functions and classes defined in other modules to run combox) or it
was not clear how to write unit tests it (the \verb+combox.gui+
contains just the \verb+ComboxConfigDialog+ a graphical front-end
which uses the configuration function defined in the
\verb+combox.config+ module to complete the combox configuration based
on the user input).

It must be noted here that pure Test Driven Development (TDD) was not
observed -- most of the time the function/class was written before the
its corresponding test was written.

\subsection{Benefits}

While writing unit tests definitely increased the time to write a
particular feature, it enabled me to immediately check if a feature
worked as it should for the given use case or given set of inputs.

With the benefit of hindsight, unit tests greatly helped in testing
the compatibility of combox on OSX. Before the \verb+v0.1.0+ release,
combox's node directory monitor always assumed that a file's first
shard (\verb+shard0+) is always available; while this assumption did
not create any problems on GNU/Linux, on OS X, this assumption made
the node directory monitor to behave erraticly -- this issue (bug
\#4\cite{combox-issue-tracker} was immediately found when the unit
tests were run for the first time on OS X. Another instance where unit
tests helped was just before the \verb+v0.2.0+ release; major changes,
including the introduction of file locks in the
\verb+ComboxDirMonitor+, were made to the \verb+combox.events+. When
the unit tests were run OS X, two tests failed, revealing a difference
in behaviour of watchdog\cite{pylib:watchdog} on GNU/Linux and OS X on
file creation\cite{combox-wd-fix}; without unit tests, there is a high
probability that this bug would never have been found by now.

\subsection{Caveats}

Unit tests are helpful in testing the correctness of a feature for
\verb+N+ number of use cases but it does not necessarily mean the
written feature correctly behaves for use cases that the author of the
feature did not consider or did not think about while writing the
respective feature. As Dijkstra correctly observed\cite{dijkstra69}:

\begin{quote}
Testing shows the presence, not the absence of bugs
\end{quote}

Unit tests failed reveal bugs \#4, \#5 \#6 \#7 \#5 \#10
\#11\cite{combox-issue-tracker}; these bugs were found when manually
testing combox.

\section{Manual testing}

\section{Stress testing}

\begin{figure}[h]
\centering
% GNUPLOT: LaTeX picture
\setlength{\unitlength}{0.240900pt}
\ifx\plotpoint\undefined\newsavebox{\plotpoint}\fi
\sbox{\plotpoint}{\rule[-0.200pt]{0.400pt}{0.400pt}}%
\begin{picture}(1535,944)(0,0)
\sbox{\plotpoint}{\rule[-0.200pt]{0.400pt}{0.400pt}}%
\put(171.0,131.0){\rule[-0.200pt]{313.893pt}{0.400pt}}
\put(171.0,131.0){\rule[-0.200pt]{4.818pt}{0.400pt}}
\put(151,131){\makebox(0,0)[r]{0}}
\put(1454.0,131.0){\rule[-0.200pt]{4.818pt}{0.400pt}}
\put(171.0,260.0){\rule[-0.200pt]{313.893pt}{0.400pt}}
\put(171.0,260.0){\rule[-0.200pt]{4.818pt}{0.400pt}}
\put(151,260){\makebox(0,0)[r]{200}}
\put(1454.0,260.0){\rule[-0.200pt]{4.818pt}{0.400pt}}
\put(171.0,388.0){\rule[-0.200pt]{313.893pt}{0.400pt}}
\put(171.0,388.0){\rule[-0.200pt]{4.818pt}{0.400pt}}
\put(151,388){\makebox(0,0)[r]{400}}
\put(1454.0,388.0){\rule[-0.200pt]{4.818pt}{0.400pt}}
\put(171.0,517.0){\rule[-0.200pt]{313.893pt}{0.400pt}}
\put(171.0,517.0){\rule[-0.200pt]{4.818pt}{0.400pt}}
\put(151,517){\makebox(0,0)[r]{600}}
\put(1454.0,517.0){\rule[-0.200pt]{4.818pt}{0.400pt}}
\put(171.0,646.0){\rule[-0.200pt]{313.893pt}{0.400pt}}
\put(171.0,646.0){\rule[-0.200pt]{4.818pt}{0.400pt}}
\put(151,646){\makebox(0,0)[r]{800}}
\put(1454.0,646.0){\rule[-0.200pt]{4.818pt}{0.400pt}}
\put(171.0,774.0){\rule[-0.200pt]{313.893pt}{0.400pt}}
\put(171.0,774.0){\rule[-0.200pt]{4.818pt}{0.400pt}}
\put(151,774){\makebox(0,0)[r]{1000}}
\put(1454.0,774.0){\rule[-0.200pt]{4.818pt}{0.400pt}}
\put(171.0,903.0){\rule[-0.200pt]{313.893pt}{0.400pt}}
\put(171.0,903.0){\rule[-0.200pt]{4.818pt}{0.400pt}}
\put(151,903){\makebox(0,0)[r]{1200}}
\put(1454.0,903.0){\rule[-0.200pt]{4.818pt}{0.400pt}}
\put(171.0,131.0){\rule[-0.200pt]{0.400pt}{185.975pt}}
\put(171.0,131.0){\rule[-0.200pt]{0.400pt}{4.818pt}}
\put(171,90){\makebox(0,0){0}}
\put(171.0,883.0){\rule[-0.200pt]{0.400pt}{4.818pt}}
\put(388.0,131.0){\rule[-0.200pt]{0.400pt}{171.280pt}}
\put(388.0,883.0){\rule[-0.200pt]{0.400pt}{4.818pt}}
\put(388.0,131.0){\rule[-0.200pt]{0.400pt}{4.818pt}}
\put(388,90){\makebox(0,0){2000}}
\put(388.0,883.0){\rule[-0.200pt]{0.400pt}{4.818pt}}
\put(605.0,131.0){\rule[-0.200pt]{0.400pt}{171.280pt}}
\put(605.0,883.0){\rule[-0.200pt]{0.400pt}{4.818pt}}
\put(605.0,131.0){\rule[-0.200pt]{0.400pt}{4.818pt}}
\put(605,90){\makebox(0,0){4000}}
\put(605.0,883.0){\rule[-0.200pt]{0.400pt}{4.818pt}}
\put(823.0,131.0){\rule[-0.200pt]{0.400pt}{171.280pt}}
\put(823.0,883.0){\rule[-0.200pt]{0.400pt}{4.818pt}}
\put(823.0,131.0){\rule[-0.200pt]{0.400pt}{4.818pt}}
\put(823,90){\makebox(0,0){6000}}
\put(823.0,883.0){\rule[-0.200pt]{0.400pt}{4.818pt}}
\put(1040.0,131.0){\rule[-0.200pt]{0.400pt}{185.975pt}}
\put(1040.0,131.0){\rule[-0.200pt]{0.400pt}{4.818pt}}
\put(1040,90){\makebox(0,0){8000}}
\put(1040.0,883.0){\rule[-0.200pt]{0.400pt}{4.818pt}}
\put(1257.0,131.0){\rule[-0.200pt]{0.400pt}{185.975pt}}
\put(1257.0,131.0){\rule[-0.200pt]{0.400pt}{4.818pt}}
\put(1257,90){\makebox(0,0){10000}}
\put(1257.0,883.0){\rule[-0.200pt]{0.400pt}{4.818pt}}
\put(1474.0,131.0){\rule[-0.200pt]{0.400pt}{185.975pt}}
\put(1474.0,131.0){\rule[-0.200pt]{0.400pt}{4.818pt}}
\put(1474,90){\makebox(0,0){12000}}
\put(1474.0,883.0){\rule[-0.200pt]{0.400pt}{4.818pt}}
\put(171.0,131.0){\rule[-0.200pt]{0.400pt}{185.975pt}}
\put(171.0,131.0){\rule[-0.200pt]{313.893pt}{0.400pt}}
\put(1474.0,131.0){\rule[-0.200pt]{0.400pt}{185.975pt}}
\put(171.0,903.0){\rule[-0.200pt]{313.893pt}{0.400pt}}
\put(30,599){\makebox(0,0){time taken (s)}}
\put(822,29){\makebox(0,0){data processed (MiB)}}
\put(691,862){\makebox(0,0)[r]{time to process all files}}
\put(711.0,862.0){\rule[-0.200pt]{24.090pt}{0.400pt}}
\put(217,159){\usebox{\plotpoint}}
\multiput(217.00,159.58)(1.034,0.499){123}{\rule{0.925pt}{0.120pt}}
\multiput(217.00,158.17)(128.079,63.000){2}{\rule{0.463pt}{0.400pt}}
\multiput(347.00,222.58)(0.892,0.500){523}{\rule{0.813pt}{0.120pt}}
\multiput(347.00,221.17)(467.312,263.000){2}{\rule{0.407pt}{0.400pt}}
\multiput(816.00,485.58)(0.770,0.500){683}{\rule{0.716pt}{0.120pt}}
\multiput(816.00,484.17)(526.514,343.000){2}{\rule{0.358pt}{0.400pt}}
\put(217,159){\makebox(0,0){$+$}}
\put(347,222){\makebox(0,0){$+$}}
\put(816,485){\makebox(0,0){$+$}}
\put(1344,828){\makebox(0,0){$+$}}
\put(761,862){\makebox(0,0){$+$}}
\put(171.0,131.0){\rule[-0.200pt]{0.400pt}{185.975pt}}
\put(171.0,131.0){\rule[-0.200pt]{313.893pt}{0.400pt}}
\put(1474.0,131.0){\rule[-0.200pt]{0.400pt}{185.975pt}}
\put(171.0,903.0){\rule[-0.200pt]{313.893pt}{0.400pt}}
\end{picture}

\caption{time to process all files}
\end{figure}

\begin{figure}[h]
\centering
% GNUPLOT: LaTeX picture
\setlength{\unitlength}{0.240900pt}
\ifx\plotpoint\undefined\newsavebox{\plotpoint}\fi
\begin{picture}(1535,944)(0,0)
\sbox{\plotpoint}{\rule[-0.200pt]{0.400pt}{0.400pt}}%
\put(281.0,131.0){\rule[-0.200pt]{287.394pt}{0.400pt}}
\put(281.0,131.0){\rule[-0.200pt]{4.818pt}{0.400pt}}
\put(261,131){\makebox(0,0)[r]{0}}
\put(1454.0,131.0){\rule[-0.200pt]{4.818pt}{0.400pt}}
\put(281.0,241.0){\rule[-0.200pt]{287.394pt}{0.400pt}}
\put(281.0,241.0){\rule[-0.200pt]{4.818pt}{0.400pt}}
\put(261,241){\makebox(0,0)[r]{500}}
\put(1454.0,241.0){\rule[-0.200pt]{4.818pt}{0.400pt}}
\put(281.0,352.0){\rule[-0.200pt]{287.394pt}{0.400pt}}
\put(281.0,352.0){\rule[-0.200pt]{4.818pt}{0.400pt}}
\put(261,352){\makebox(0,0)[r]{1000}}
\put(1454.0,352.0){\rule[-0.200pt]{4.818pt}{0.400pt}}
\put(281.0,462.0){\rule[-0.200pt]{287.394pt}{0.400pt}}
\put(281.0,462.0){\rule[-0.200pt]{4.818pt}{0.400pt}}
\put(261,462){\makebox(0,0)[r]{1500}}
\put(1454.0,462.0){\rule[-0.200pt]{4.818pt}{0.400pt}}
\put(281.0,572.0){\rule[-0.200pt]{287.394pt}{0.400pt}}
\put(281.0,572.0){\rule[-0.200pt]{4.818pt}{0.400pt}}
\put(261,572){\makebox(0,0)[r]{2000}}
\put(1454.0,572.0){\rule[-0.200pt]{4.818pt}{0.400pt}}
\put(281.0,682.0){\rule[-0.200pt]{287.394pt}{0.400pt}}
\put(281.0,682.0){\rule[-0.200pt]{4.818pt}{0.400pt}}
\put(261,682){\makebox(0,0)[r]{2500}}
\put(1454.0,682.0){\rule[-0.200pt]{4.818pt}{0.400pt}}
\put(281.0,793.0){\rule[-0.200pt]{287.394pt}{0.400pt}}
\put(281.0,793.0){\rule[-0.200pt]{4.818pt}{0.400pt}}
\put(261,793){\makebox(0,0)[r]{3000}}
\put(1454.0,793.0){\rule[-0.200pt]{4.818pt}{0.400pt}}
\put(281.0,903.0){\rule[-0.200pt]{287.394pt}{0.400pt}}
\put(281.0,903.0){\rule[-0.200pt]{4.818pt}{0.400pt}}
\put(261,903){\makebox(0,0)[r]{3500}}
\put(1454.0,903.0){\rule[-0.200pt]{4.818pt}{0.400pt}}
\put(281.0,131.0){\rule[-0.200pt]{0.400pt}{185.975pt}}
\put(281.0,131.0){\rule[-0.200pt]{0.400pt}{4.818pt}}
\put(281,90){\makebox(0,0){0}}
\put(281.0,883.0){\rule[-0.200pt]{0.400pt}{4.818pt}}
\put(480.0,131.0){\rule[-0.200pt]{0.400pt}{171.280pt}}
\put(480.0,883.0){\rule[-0.200pt]{0.400pt}{4.818pt}}
\put(480.0,131.0){\rule[-0.200pt]{0.400pt}{4.818pt}}
\put(480,90){\makebox(0,0){2000}}
\put(480.0,883.0){\rule[-0.200pt]{0.400pt}{4.818pt}}
\put(679.0,131.0){\rule[-0.200pt]{0.400pt}{171.280pt}}
\put(679.0,883.0){\rule[-0.200pt]{0.400pt}{4.818pt}}
\put(679.0,131.0){\rule[-0.200pt]{0.400pt}{4.818pt}}
\put(679,90){\makebox(0,0){4000}}
\put(679.0,883.0){\rule[-0.200pt]{0.400pt}{4.818pt}}
\put(878.0,131.0){\rule[-0.200pt]{0.400pt}{171.280pt}}
\put(878.0,883.0){\rule[-0.200pt]{0.400pt}{4.818pt}}
\put(878.0,131.0){\rule[-0.200pt]{0.400pt}{4.818pt}}
\put(878,90){\makebox(0,0){6000}}
\put(878.0,883.0){\rule[-0.200pt]{0.400pt}{4.818pt}}
\put(1076.0,131.0){\rule[-0.200pt]{0.400pt}{171.280pt}}
\put(1076.0,883.0){\rule[-0.200pt]{0.400pt}{4.818pt}}
\put(1076.0,131.0){\rule[-0.200pt]{0.400pt}{4.818pt}}
\put(1076,90){\makebox(0,0){8000}}
\put(1076.0,883.0){\rule[-0.200pt]{0.400pt}{4.818pt}}
\put(1275.0,131.0){\rule[-0.200pt]{0.400pt}{185.975pt}}
\put(1275.0,131.0){\rule[-0.200pt]{0.400pt}{4.818pt}}
\put(1275,90){\makebox(0,0){10000}}
\put(1275.0,883.0){\rule[-0.200pt]{0.400pt}{4.818pt}}
\put(1474.0,131.0){\rule[-0.200pt]{0.400pt}{185.975pt}}
\put(1474.0,131.0){\rule[-0.200pt]{0.400pt}{4.818pt}}
\put(1474,90){\makebox(0,0){12000}}
\put(1474.0,883.0){\rule[-0.200pt]{0.400pt}{4.818pt}}
\put(281.0,131.0){\rule[-0.200pt]{0.400pt}{185.975pt}}
\put(281.0,131.0){\rule[-0.200pt]{287.394pt}{0.400pt}}
\put(1474.0,131.0){\rule[-0.200pt]{0.400pt}{185.975pt}}
\put(281.0,903.0){\rule[-0.200pt]{287.394pt}{0.400pt}}
\put(30,631){\makebox(0,0){avg time taken (ms)}}
\put(877,29){\makebox(0,0){total size of files processed (MiB)}}
\put(981,862){\makebox(0,0)[r]{avg. time to split \& encrypt file}}
\put(1001.0,862.0){\rule[-0.200pt]{24.090pt}{0.400pt}}
\put(323,209){\usebox{\plotpoint}}
\multiput(323.58,209.00)(0.499,2.152){235}{\rule{0.120pt}{1.818pt}}
\multiput(322.17,209.00)(119.000,507.227){2}{\rule{0.400pt}{0.909pt}}
\multiput(442.00,720.58)(3.175,0.499){133}{\rule{2.629pt}{0.120pt}}
\multiput(442.00,719.17)(424.543,68.000){2}{\rule{1.315pt}{0.400pt}}
\multiput(872.00,788.58)(2.471,0.499){193}{\rule{2.071pt}{0.120pt}}
\multiput(872.00,787.17)(478.701,98.000){2}{\rule{1.036pt}{0.400pt}}
\put(323,209){\makebox(0,0){$+$}}
\put(442,720){\makebox(0,0){$+$}}
\put(872,788){\makebox(0,0){$+$}}
\put(1355,886){\makebox(0,0){$+$}}
\put(1051,862){\makebox(0,0){$+$}}
\put(281.0,131.0){\rule[-0.200pt]{0.400pt}{185.975pt}}
\put(281.0,131.0){\rule[-0.200pt]{287.394pt}{0.400pt}}
\put(1474.0,131.0){\rule[-0.200pt]{0.400pt}{185.975pt}}
\put(281.0,903.0){\rule[-0.200pt]{287.394pt}{0.400pt}}
\end{picture}

\caption{avg. time to split and encrypt}
\end{figure}

\begin{figure}[h]
\centering
% GNUPLOT: LaTeX picture
\setlength{\unitlength}{0.240900pt}
\ifx\plotpoint\undefined\newsavebox{\plotpoint}\fi
\begin{picture}(1535,944)(0,0)
\sbox{\plotpoint}{\rule[-0.200pt]{0.400pt}{0.400pt}}%
\put(211.0,131.0){\rule[-0.200pt]{304.257pt}{0.400pt}}
\put(211.0,131.0){\rule[-0.200pt]{4.818pt}{0.400pt}}
\put(191,131){\makebox(0,0)[r]{0}}
\put(1454.0,131.0){\rule[-0.200pt]{4.818pt}{0.400pt}}
\put(211.0,260.0){\rule[-0.200pt]{304.257pt}{0.400pt}}
\put(211.0,260.0){\rule[-0.200pt]{4.818pt}{0.400pt}}
\put(191,260){\makebox(0,0)[r]{200}}
\put(1454.0,260.0){\rule[-0.200pt]{4.818pt}{0.400pt}}
\put(211.0,388.0){\rule[-0.200pt]{304.257pt}{0.400pt}}
\put(211.0,388.0){\rule[-0.200pt]{4.818pt}{0.400pt}}
\put(191,388){\makebox(0,0)[r]{400}}
\put(1454.0,388.0){\rule[-0.200pt]{4.818pt}{0.400pt}}
\put(211.0,517.0){\rule[-0.200pt]{304.257pt}{0.400pt}}
\put(211.0,517.0){\rule[-0.200pt]{4.818pt}{0.400pt}}
\put(191,517){\makebox(0,0)[r]{600}}
\put(1454.0,517.0){\rule[-0.200pt]{4.818pt}{0.400pt}}
\put(211.0,646.0){\rule[-0.200pt]{304.257pt}{0.400pt}}
\put(211.0,646.0){\rule[-0.200pt]{4.818pt}{0.400pt}}
\put(191,646){\makebox(0,0)[r]{800}}
\put(1454.0,646.0){\rule[-0.200pt]{4.818pt}{0.400pt}}
\put(211.0,774.0){\rule[-0.200pt]{304.257pt}{0.400pt}}
\put(211.0,774.0){\rule[-0.200pt]{4.818pt}{0.400pt}}
\put(191,774){\makebox(0,0)[r]{1000}}
\put(1454.0,774.0){\rule[-0.200pt]{4.818pt}{0.400pt}}
\put(211.0,903.0){\rule[-0.200pt]{304.257pt}{0.400pt}}
\put(211.0,903.0){\rule[-0.200pt]{4.818pt}{0.400pt}}
\put(191,903){\makebox(0,0)[r]{1200}}
\put(1454.0,903.0){\rule[-0.200pt]{4.818pt}{0.400pt}}
\put(211.0,131.0){\rule[-0.200pt]{0.400pt}{185.975pt}}
\put(211.0,131.0){\rule[-0.200pt]{0.400pt}{4.818pt}}
\put(211,90){\makebox(0,0){0}}
\put(211.0,883.0){\rule[-0.200pt]{0.400pt}{4.818pt}}
\put(422.0,131.0){\rule[-0.200pt]{0.400pt}{161.403pt}}
\put(422.0,883.0){\rule[-0.200pt]{0.400pt}{4.818pt}}
\put(422.0,131.0){\rule[-0.200pt]{0.400pt}{4.818pt}}
\put(422,90){\makebox(0,0){2000}}
\put(422.0,883.0){\rule[-0.200pt]{0.400pt}{4.818pt}}
\put(632.0,131.0){\rule[-0.200pt]{0.400pt}{161.403pt}}
\put(632.0,883.0){\rule[-0.200pt]{0.400pt}{4.818pt}}
\put(632.0,131.0){\rule[-0.200pt]{0.400pt}{4.818pt}}
\put(632,90){\makebox(0,0){4000}}
\put(632.0,883.0){\rule[-0.200pt]{0.400pt}{4.818pt}}
\put(843.0,131.0){\rule[-0.200pt]{0.400pt}{161.403pt}}
\put(843.0,883.0){\rule[-0.200pt]{0.400pt}{4.818pt}}
\put(843.0,131.0){\rule[-0.200pt]{0.400pt}{4.818pt}}
\put(843,90){\makebox(0,0){6000}}
\put(843.0,883.0){\rule[-0.200pt]{0.400pt}{4.818pt}}
\put(1053.0,131.0){\rule[-0.200pt]{0.400pt}{185.975pt}}
\put(1053.0,131.0){\rule[-0.200pt]{0.400pt}{4.818pt}}
\put(1053,90){\makebox(0,0){8000}}
\put(1053.0,883.0){\rule[-0.200pt]{0.400pt}{4.818pt}}
\put(1264.0,131.0){\rule[-0.200pt]{0.400pt}{185.975pt}}
\put(1264.0,131.0){\rule[-0.200pt]{0.400pt}{4.818pt}}
\put(1264,90){\makebox(0,0){10000}}
\put(1264.0,883.0){\rule[-0.200pt]{0.400pt}{4.818pt}}
\put(1474.0,131.0){\rule[-0.200pt]{0.400pt}{185.975pt}}
\put(1474.0,131.0){\rule[-0.200pt]{0.400pt}{4.818pt}}
\put(1474,90){\makebox(0,0){12000}}
\put(1474.0,883.0){\rule[-0.200pt]{0.400pt}{4.818pt}}
\put(211.0,131.0){\rule[-0.200pt]{0.400pt}{185.975pt}}
\put(211.0,131.0){\rule[-0.200pt]{304.257pt}{0.400pt}}
\put(1474.0,131.0){\rule[-0.200pt]{0.400pt}{185.975pt}}
\put(211.0,903.0){\rule[-0.200pt]{304.257pt}{0.400pt}}
\put(30,586){\makebox(0,0){time taken (s)}}
\put(842,29){\makebox(0,0){data processed (MiB)}}
\put(871,862){\makebox(0,0)[r]{time to process all files (2016)}}
\put(891.0,862.0){\rule[-0.200pt]{24.090pt}{0.400pt}}
\put(256,159){\usebox{\plotpoint}}
\multiput(256.00,159.58)(1.002,0.499){123}{\rule{0.900pt}{0.120pt}}
\multiput(256.00,158.17)(124.132,63.000){2}{\rule{0.450pt}{0.400pt}}
\multiput(382.00,222.58)(0.863,0.500){523}{\rule{0.790pt}{0.120pt}}
\multiput(382.00,221.17)(452.359,263.000){2}{\rule{0.395pt}{0.400pt}}
\multiput(836.00,485.58)(0.746,0.500){683}{\rule{0.697pt}{0.120pt}}
\multiput(836.00,484.17)(510.553,343.000){2}{\rule{0.349pt}{0.400pt}}
\put(256,159){\makebox(0,0){$+$}}
\put(382,222){\makebox(0,0){$+$}}
\put(836,485){\makebox(0,0){$+$}}
\put(1348,828){\makebox(0,0){$+$}}
\put(941,862){\makebox(0,0){$+$}}
\put(871,821){\makebox(0,0)[r]{time to process all files (2015)}}
\multiput(891,821)(20.756,0.000){5}{\usebox{\plotpoint}}
\put(991,821){\usebox{\plotpoint}}
\put(256,200){\usebox{\plotpoint}}
\multiput(256,200)(20.499,3.254){7}{\usebox{\plotpoint}}
\multiput(382,220)(17.264,11.522){26}{\usebox{\plotpoint}}
\multiput(836,523)(18.433,9.540){28}{\usebox{\plotpoint}}
\put(1348,788){\usebox{\plotpoint}}
\put(256,200){\makebox(0,0){$\times$}}
\put(382,220){\makebox(0,0){$\times$}}
\put(836,523){\makebox(0,0){$\times$}}
\put(1348,788){\makebox(0,0){$\times$}}
\put(941,821){\makebox(0,0){$\times$}}
\put(211.0,131.0){\rule[-0.200pt]{0.400pt}{185.975pt}}
\put(211.0,131.0){\rule[-0.200pt]{304.257pt}{0.400pt}}
\put(1474.0,131.0){\rule[-0.200pt]{0.400pt}{185.975pt}}
\put(211.0,903.0){\rule[-0.200pt]{304.257pt}{0.400pt}}
\end{picture}

\caption{time to process all files - difference between 2015 and 2016}
\end{figure}

\begin{figure}[h]
\centering
% GNUPLOT: LaTeX picture
\setlength{\unitlength}{0.240900pt}
\ifx\plotpoint\undefined\newsavebox{\plotpoint}\fi
\begin{picture}(1535,944)(0,0)
\sbox{\plotpoint}{\rule[-0.200pt]{0.400pt}{0.400pt}}%
\put(281.0,131.0){\rule[-0.200pt]{287.394pt}{0.400pt}}
\put(281.0,131.0){\rule[-0.200pt]{4.818pt}{0.400pt}}
\put(261,131){\makebox(0,0)[r]{0}}
\put(1454.0,131.0){\rule[-0.200pt]{4.818pt}{0.400pt}}
\put(281.0,241.0){\rule[-0.200pt]{287.394pt}{0.400pt}}
\put(281.0,241.0){\rule[-0.200pt]{4.818pt}{0.400pt}}
\put(261,241){\makebox(0,0)[r]{500}}
\put(1454.0,241.0){\rule[-0.200pt]{4.818pt}{0.400pt}}
\put(281.0,352.0){\rule[-0.200pt]{287.394pt}{0.400pt}}
\put(281.0,352.0){\rule[-0.200pt]{4.818pt}{0.400pt}}
\put(261,352){\makebox(0,0)[r]{1000}}
\put(1454.0,352.0){\rule[-0.200pt]{4.818pt}{0.400pt}}
\put(281.0,462.0){\rule[-0.200pt]{287.394pt}{0.400pt}}
\put(281.0,462.0){\rule[-0.200pt]{4.818pt}{0.400pt}}
\put(261,462){\makebox(0,0)[r]{1500}}
\put(1454.0,462.0){\rule[-0.200pt]{4.818pt}{0.400pt}}
\put(281.0,572.0){\rule[-0.200pt]{287.394pt}{0.400pt}}
\put(281.0,572.0){\rule[-0.200pt]{4.818pt}{0.400pt}}
\put(261,572){\makebox(0,0)[r]{2000}}
\put(1454.0,572.0){\rule[-0.200pt]{4.818pt}{0.400pt}}
\put(281.0,682.0){\rule[-0.200pt]{287.394pt}{0.400pt}}
\put(281.0,682.0){\rule[-0.200pt]{4.818pt}{0.400pt}}
\put(261,682){\makebox(0,0)[r]{2500}}
\put(1454.0,682.0){\rule[-0.200pt]{4.818pt}{0.400pt}}
\put(281.0,793.0){\rule[-0.200pt]{287.394pt}{0.400pt}}
\put(281.0,793.0){\rule[-0.200pt]{4.818pt}{0.400pt}}
\put(261,793){\makebox(0,0)[r]{3000}}
\put(1454.0,793.0){\rule[-0.200pt]{4.818pt}{0.400pt}}
\put(281.0,903.0){\rule[-0.200pt]{287.394pt}{0.400pt}}
\put(281.0,903.0){\rule[-0.200pt]{4.818pt}{0.400pt}}
\put(261,903){\makebox(0,0)[r]{3500}}
\put(1454.0,903.0){\rule[-0.200pt]{4.818pt}{0.400pt}}
\put(281.0,131.0){\rule[-0.200pt]{0.400pt}{185.975pt}}
\put(281.0,131.0){\rule[-0.200pt]{0.400pt}{4.818pt}}
\put(281,90){\makebox(0,0){0}}
\put(281.0,883.0){\rule[-0.200pt]{0.400pt}{4.818pt}}
\put(480.0,131.0){\rule[-0.200pt]{0.400pt}{185.975pt}}
\put(480.0,131.0){\rule[-0.200pt]{0.400pt}{4.818pt}}
\put(480,90){\makebox(0,0){2000}}
\put(480.0,883.0){\rule[-0.200pt]{0.400pt}{4.818pt}}
\put(679.0,131.0){\rule[-0.200pt]{0.400pt}{83.110pt}}
\put(679.0,558.0){\rule[-0.200pt]{0.400pt}{83.110pt}}
\put(679.0,131.0){\rule[-0.200pt]{0.400pt}{4.818pt}}
\put(679,90){\makebox(0,0){4000}}
\put(679.0,883.0){\rule[-0.200pt]{0.400pt}{4.818pt}}
\put(878.0,131.0){\rule[-0.200pt]{0.400pt}{83.110pt}}
\put(878.0,558.0){\rule[-0.200pt]{0.400pt}{83.110pt}}
\put(878.0,131.0){\rule[-0.200pt]{0.400pt}{4.818pt}}
\put(878,90){\makebox(0,0){6000}}
\put(878.0,883.0){\rule[-0.200pt]{0.400pt}{4.818pt}}
\put(1076.0,131.0){\rule[-0.200pt]{0.400pt}{83.110pt}}
\put(1076.0,558.0){\rule[-0.200pt]{0.400pt}{83.110pt}}
\put(1076.0,131.0){\rule[-0.200pt]{0.400pt}{4.818pt}}
\put(1076,90){\makebox(0,0){8000}}
\put(1076.0,883.0){\rule[-0.200pt]{0.400pt}{4.818pt}}
\put(1275.0,131.0){\rule[-0.200pt]{0.400pt}{83.110pt}}
\put(1275.0,558.0){\rule[-0.200pt]{0.400pt}{83.110pt}}
\put(1275.0,131.0){\rule[-0.200pt]{0.400pt}{4.818pt}}
\put(1275,90){\makebox(0,0){10000}}
\put(1275.0,883.0){\rule[-0.200pt]{0.400pt}{4.818pt}}
\put(1474.0,131.0){\rule[-0.200pt]{0.400pt}{185.975pt}}
\put(1474.0,131.0){\rule[-0.200pt]{0.400pt}{4.818pt}}
\put(1474,90){\makebox(0,0){12000}}
\put(1474.0,883.0){\rule[-0.200pt]{0.400pt}{4.818pt}}
\put(281.0,131.0){\rule[-0.200pt]{0.400pt}{185.975pt}}
\put(281.0,131.0){\rule[-0.200pt]{287.394pt}{0.400pt}}
\put(1474.0,131.0){\rule[-0.200pt]{0.400pt}{185.975pt}}
\put(281.0,903.0){\rule[-0.200pt]{287.394pt}{0.400pt}}
\put(30,631){\makebox(0,0){avg time taken (ms)}}
\put(877,29){\makebox(0,0){total size of files processed (MiB)}}
\put(1314,537){\makebox(0,0)[r]{avg. time to split \& encrypt file (2016)}}
\put(1334.0,537.0){\rule[-0.200pt]{24.090pt}{0.400pt}}
\put(323,209){\usebox{\plotpoint}}
\multiput(323.58,209.00)(0.499,2.152){235}{\rule{0.120pt}{1.818pt}}
\multiput(322.17,209.00)(119.000,507.227){2}{\rule{0.400pt}{0.909pt}}
\multiput(442.00,720.58)(3.175,0.499){133}{\rule{2.629pt}{0.120pt}}
\multiput(442.00,719.17)(424.543,68.000){2}{\rule{1.315pt}{0.400pt}}
\multiput(872.00,788.58)(2.471,0.499){193}{\rule{2.071pt}{0.120pt}}
\multiput(872.00,787.17)(478.701,98.000){2}{\rule{1.036pt}{0.400pt}}
\put(323,209){\makebox(0,0){$+$}}
\put(442,720){\makebox(0,0){$+$}}
\put(872,788){\makebox(0,0){$+$}}
\put(1355,886){\makebox(0,0){$+$}}
\put(1384,537){\makebox(0,0){$+$}}
\put(1314,496){\makebox(0,0)[r]{avg. time to split \& encrypt file (2015)}}
\multiput(1334,496)(20.756,0.000){5}{\usebox{\plotpoint}}
\put(1434,496){\usebox{\plotpoint}}
\put(323,215){\usebox{\plotpoint}}
\multiput(323,215)(4.708,20.215){26}{\usebox{\plotpoint}}
\multiput(442,726)(20.739,0.820){21}{\usebox{\plotpoint}}
\multiput(872,743)(20.622,2.348){23}{\usebox{\plotpoint}}
\put(1355,798){\usebox{\plotpoint}}
\put(323,215){\makebox(0,0){$\times$}}
\put(442,726){\makebox(0,0){$\times$}}
\put(872,743){\makebox(0,0){$\times$}}
\put(1355,798){\makebox(0,0){$\times$}}
\put(1384,496){\makebox(0,0){$\times$}}
\put(281.0,131.0){\rule[-0.200pt]{0.400pt}{185.975pt}}
\put(281.0,131.0){\rule[-0.200pt]{287.394pt}{0.400pt}}
\put(1474.0,131.0){\rule[-0.200pt]{0.400pt}{185.975pt}}
\put(281.0,903.0){\rule[-0.200pt]{287.394pt}{0.400pt}}
\end{picture}

\caption{avg. time to split and encrypt - difference between 2015 and 2016}
\end{figure}