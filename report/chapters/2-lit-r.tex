\chapter{Background and Literature Review}

\epigraph{Books serve to show a man that those original thoughts of
  his aren't very new after all}{\textit{Abraham Lincoln}}

The idea of unifying the storage provided by multiple Internet file
storage providers and storing all the content in an encrypted form is
not new, computer researchers/scientists, programmers have devised
different methods to use multiple file storage providers' storage
space. This chapter gives an overview of the work done by Yeo et
al. in unifying the storage provided by Dropbox, Box, Google Drive and
Skydrive on Android devices\cite{yeo}(Section \ref{3-yeo-sec});
SkyCDS, a content delivery service, by Gonzalez et al., which uses
publish/subscribe overly paradigm and stores the content across
multiple ``cloud'' storage providers such that only part of the
content (in encrypted form) is stored on each ``cloud'' storage
provider\cite{skycds}(Section \ref{3-skycds-sec}); lastly,
\verb+git-annex+, by Joey Hess\cite{person:joeyh}, that allows one to
version control and keep track of large files with a possibility of
encrypting files that are stored in ``special remotes'' -- storage
provided by Internet file storage providers (Section
\ref{3-gitannex-sec}).

\section{Multi Cloud Storage Prototype}\label{3-yeo-sec}

In their paper ``Leveraging client-side storage techniques for
enhanced use of multiple consumer cloud storage services on
resource-constrained mobile devices'', Yeo et al. show their Android
mobile application, a prototype, which unifies storage provided by
Dropbox, Box, Google Drive and SkyDrive. The application allows the
user to store all their information in a single location on their
phone and the application uses erasure coding\cite{weatherspoon} to
split each file into \verb`n + k` fragments and spreads the encrypted
fragments across storage provided by the file storage providers. All
basic file operations -- Create, Rename, Update, Delete (CRUD) -- are
possible. Information about the file stored in a unified location is
stored in a SQLite database. Unlike combox, which depends the file
storage provider' client to sync file fragments/shards to the file
storage provider's server, the android application developed by Yeo et
al. takes the responsibility to sync file fragments/shards to each
file storage provider and usesd the OAuth 2.0\cite{protocal:oauth2}
protocol for authorization.

For encrypting file fragments, they use AES-256; they key for
encrypting is derived from the user's password by using Password-Based
Key Derivation Function (PBKDF2)\cite{kaliski}. For erasure coding
they use the JigDFS librarary\cite{jigdfs}. The android application is
able do ``progressive streaming'' of media files; this means that
large media files can be streamed in real-time from the from the file
storage providers' servers; this is an attractive feature in a
``resource contrained'' device where storage is expensive.

Yeo et al. propose methods for achieving data de-duplication, file
fragment/shard compression based on the type of the file, intelligent
pre-fetching and caching for file fragrments and ``automatic
restoration in exploiting file-versioning''; these features were not
implemented in the prototype Android application and there is
possibility of Yeo et al. implementing these features in the future.

It becomes that that Yeo et al. work is of immense importance when we
take into consideration the research done by Yang et al., which found
that 59\% of the users who use ``cloud storage service'' access the
service through a smart phone and 42.2\% users access
audio/video\cite{yang}. The research by Yang et al. definitely
suggests a trend of users' preference for small hand-held computers
over laptops and desktops.

\section{SkyCDS}\label{3-skycds-sec}

SkyCDS, by Gonzalez et al., is a content delivery system that splits
and spreads the content across multiple ``cloud'' storage
providers\cite{skycds}. According to Gonzalez et al., the main reason
for designing and developing SkyCDS was to prevent content providers
from getting locked into just one ``cloud'' storage provider and to
minimize loss when a ``cloud'' storage provider goes out of business
or if there is temporary outage in the storage service provided by the
``cloud'' storage provider.

In SkyCDS the content delivery to subscribers of the content is
segregated into two distinct layers -- Metadata Flow Layer and the
Content Flow Layer. The publisher of the content largely interacts
with the Metadata Flow Layer that controls and keeps track of the what
content is published and the subscriber also largely interacts with
the Metadata Flow layer to subscribe to content published in the
content delivery system. The Content Flow Layer is where the content
is stored across multiple ``cloud'' storage providers. The publisher
is responsible for publishing the content using eth ``delivery
workflow'' (part of the Content Flow Layer) and the subscriber uses
the ``retrieve workflow'' to get access to the subscribed content.

When content has to be dispersed to $k$ ``cloud'' storage providers,
the content is split into $n$ chunks, $n > k$, this file splitting
seems to produce 66.7\% of redundancy overhead\cite{skycds}; this file
splitting scheme looks very similar to erasure coding, but Gonzalez et
al. don't explicitly state that the content splitting scheme is indeed
``erasure coding''. The splitting of content is done by the ``delivery
workflow'' engine which is invoked when the publisher triggers the
action to publish the respective content to subscribers.

To evaluate the effectiveness of SkyCDS, Gonzalez et al. state that
they've done a case study using the data (content) obtained from
European Space Astronomy Center (ESAC) for the Soil Moisture Ocean
Salinity. In this study, a group of organizations, in two different
continents, used SkyCDS to share satillete images with each
other. According to Gonzalez et al. this study attested SkyCDS as a
viable option for content delivery with respective to performance,
cost of ``cloud'' storage space and reliability.

\section{git-annex}\label{3-gitannex-sec}

\verb+git-annex+ allows one to version controlled large files that are
not usually feasible to version control under
\verb+git+\cite{program:git}. \verb+git-annex+, checks in the names
and other meta-data about the files in git and stores the actual
content under \verb+.git/annex+ directory. When a file is added to
\verb+git-annex+, a symlink of the file is created in place of th file
and the content of the file itself is stored under the
\verb+.git/annex+ directory.

For instance, say there is a file called
\verb+deb-nicholson-80s.medium.webm+ was downloaded from the Internet
to the \verb+git-annex+ directory:

\begin{verbatim}
↳ git status
On branch master
Untracked files:
  (use "git add <file>..." to include in what will be committed)

   deb-nicholson-80s.medium.webm

↳ ls -l
total 105708
...
-rw-r--r-- 1 rsd rsd 108196923 May  5  2015 deb-nicholson-80s.medium.webm
...
\end{verbatim}

When this file is added to \verb+git-annex+ with \verb+git annex add+,
the file turns into a symlink to a file under the \verb+.git/annex+
directory:

{\small
\begin{verbatim}
↳ git annex add deb-nicholson-80s.medium.webm
add deb-nicholson-80s.medium.webm ok
(recording state in git...)

↳ ls -l
...
lrwxrwxrwx 1 rsd rsd   207 May  5  2015 deb-nicholson-80s.medium.webm -> ../.git/an
nex/objects/3j/vG/SHA256E-s108196923--7de9484ee96908268e21b451eb9805552c32b44da08e7
0ee861332c87352944f.webm/SHA256E-s108196923--7de9484ee96908268e21b451eb9805552c32b4
4da08e70ee861332c87352944f.webm

↳ git commit -m "Added video/deb-nicholson-80s.medium.webm"
[master efa1775] Added video/deb-nicholson-80s.medium.webm
 1 file changed, 1 insertion(+)
 create mode 120000 video/deb-nicholson-80s.medium.webm
\end{verbatim}
}

Now, the file \verb+deb-nicholson-80s.medium.webm+ is checked into
\verb+git-annex+ and we can now do a \verb+git annex sync+ to sync the
repository to other \verb+git-annex+ repositories. It must be noted
here that that when the repository is synced, the file content itself
is not transferred to the other \verb+git-annex+ repositories; only
the file's name and its meta-data that is stored in a separate git
branch called \verb+git-annex+ are
transferred\cite{documentation:git-annex-hworks}. In order to create a
copy of a given file in another git annex repository,
\verb+git annex get /path/to/filename.ext+ has to done.

\verb+git-annex+ has this feature called ``special
remotes''\cite{documentation:git-annex-sremotes}, that allows one to
push/copy data to checked into \verb+git-annex+ to storage provided by
``cloud'' storage providers. At the time of writing this report,
\verb+git-annex+ supports pushing data to the following file storage
services:

{\scriptsize
\begin{itemize}
\item Amazon S3
\item Amazon Glacier
\item Internet Archive via S3
\item Box.com
\item Google drive
\item Google Cloud Storage
\item Mega.co.nz
\item SkyDrive
\item OwnCloud
\item Flickr
\item IMAP
\item Usenet
\item chef-vault
\item hubiC
\item pCloud
\item ipfs
\item Ceph
\item Blackblaze's B2
\end{itemize}
}

All data pushed to file storage provider's servers can be optionally
encrypted using one's GPG key. For instance, to encrypt data that is
pushed to the Amazon S3 special remote, following command is
used\cite{docs:git-annex-as3}:

\begin{verbatim}
$ git annex initremote cloud type=S3 keyid=2512E3C7
initremote cloud (encryption setup with gpg key C910D9222512E3C7) (checking bucket) (creating bucket in US) (gpg) ok
$ git annex describe cloud "at Amazon's US datacenter"
describe cloud ok
\end{verbatim}

where \verb+2512E3C7+ is the id of the GPG key to use for encrypting
data pushed to the Amazon S3 special remote. It is also possible to
store each file that is pushed to the remotes as a set of chunks of
size \verb+N+, to do that we do:

\begin{verbatim}
$ git annex initremote cloud type=S3 chunk=1MiB keyid=2512E3C7
initremote cloud (encryption setup with gpg key C910D9222512E3C7) (checking bucket) (creating bucket in US) (gpg) ok
$ git annex describe cloud "at Amazon's US datacenter"
describe cloud ok
\end{verbatim}

with that each file that has to be pushed to the Amazon S3 special
remote is divided into 1MiB chunks, each chunk is encrypted using the
GPG key \verb+2512E3C7+ and the encrypted chunks are finally pushed to
the Amazon S3 remote. It is must be noted here that unlike the Multi
Cloud Storage Prototype or SkyCDS or combox, in \verb+git-annex+ when
we are using file chunking all the chunks go to the same location --
in this case, the Amazon S3 remote.
